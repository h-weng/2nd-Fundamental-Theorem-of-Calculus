\documentclass[a4paper, 12pt]{article}
\usepackage{amsmath}
\begin{document}
\title{Fundamental Theorem of Calculus}
\date{\today}
\pagestyle{empty} 
\section{Fundamental Theorem of Calculus II}
	
As stated in $Calculus(Larson)$:	\\
\\
	If a function $f$ is continuous on the open interval $I$ containing $a$, then for every $x$ in the interval
\begin{align}
	\frac{d}{dx} \int_a^b [f(t) dt] = f(x)
\end{align}
\section{Proof}
	Begin by defining $F$ as
\begin{align}
	F(x) &= \int_a^x f(t) dt
\end{align}
	Then, you can write by the definition of the derivative
\begin{align*}
	F'(x) &= lim_{\Delta x \to 0} \frac{F(x + \Delta x) - F(x)}{\Delta x} \\
		  &= lim_{\Delta x \to 0} \frac{1}{\Delta x} [\int_{a}^{x + \Delta x} f(t) dt - \int_a^x f(t) dt ] \\
		  &=  lim_{\Delta x \to 0} \frac{1}{\Delta x} [\int_{a}^{x + \Delta x} f(t) dt + \int_x^a f(t) dt ] \\
		  &=  lim_{\Delta x \to 0} \frac{1}{\Delta x} [\int_{x}^{x + \Delta x} f(t) dt]
\end{align*}

	From the Mean Value Theorem for Integrals when $\Delta x > 0$, there exists a number $c$ in the interval $[x, x + \Delta x]$ such that the integral in the expression above is equal to $f(c) \Delta x$.  Because $x \geq c \geq x + \Delta x$, it follows that $c \to x$ as $\Delta x \to 0$.
	
\begin{align*}
	F'(x) = lim_{\Delta x \to 0} [\frac{1}{\Delta x} f(c) \Delta x ] = lim_{\Delta x \to 0} f(c) = f(x)
\end{align*}

	Using the area model for definite integrals, the approximation
\begin{align}
	f(x) \Delta x \approx \int_{x}^{x + \Delta x} f(t) dt
\end{align}	

	can be viewed as the area of the height of $f(x)$ and the width $\Delta x$ is approximately equal to the area of the region between $[x, x + \Delta x]$.

\end{document}